% Created 2015-03-13 Fri 11:06
\documentclass[11pt]{article}
\usepackage[utf8]{inputenc}
\usepackage[T1]{fontenc}
\usepackage{fixltx2e}
\usepackage{graphicx}
\usepackage{longtable}
\usepackage{float}
\usepackage{wrapfig}
\usepackage{rotating}
\usepackage[normalem]{ulem}
\usepackage{amsmath}
\usepackage{textcomp}
\usepackage{marvosym}
\usepackage{wasysym}
\usepackage{amssymb}
\usepackage{hyperref}
\tolerance=1000
\usepackage{color}
\usepackage{listings}
\usepackage{amsmath,amssymb,fullpage,dsfont,setspace}
\newcommand{\bs}{\boldsymbol}
\newcommand{\attn}[1]{\textbf{***{#1}***}}
\newcommand{\src}{\attn{source}}
\setlength{\parskip}{\baselineskip}
\newcommand{\logit}{\text{logit}}
\newcommand{\T}{\intercal}
\setstretch{1.5}
\date{}
\title{M1 Writeup}
\hypersetup{
  pdfkeywords={},
  pdfsubject={},
  pdfcreator={Emacs 24.4.1 (Org mode 8.2.10)}}
\begin{document}

\maketitle
\tableofcontents


\section{Notation}
\label{sec-1}

Each location $i \in \mathcal{L}$ lies in a $R^2$ at coordinate pair
$C_i$.  The matrix $\Omega$ is a binary matrix indicating adjacent
locations.  For any two locations $i$ and $j$, $\Omega_{i,j} = 1$ if
location $j$ is adjacent to location $i$.  We define the network such
that locations are adjacenty to themselves, $\Omega_{i,i} = 1$ for all
$i$.  All networks defined will be undirected thus $\Omega_{i,j} =
  \Omega_{j,i}$ for all $i,j$.  The adjacency matrix is the defining
component of the network.  For some of the networks, the adjacency
matrix can be defined without knowing the coordinate pairs, but for
other networks the coordinate pairs are required to generate the
adjacency matrix.  Regardless of the network, we outline every step to
generate the network in its entirety.

Defining distance between to locations in a network, for most of the
following structures, is the geodesic distance where each edge has
length $1$ regardless of the coordinate values.  For the ring and
starred alley, we make some modifications to the edge lengths before
calculating the geodesic distance.  These details are covered in the
respective sections.

When defining networks, some additonal notation may be added to keep
the process clear and concise.

To model the spread of the disease, we assume that once a location
becomes infected, it remains infected thereafter.  The infection
status for location $i$ at time $t$ is $X_{i,t} = 1$ if infected and
$X_{i,t} = 0$ otherwise.  Thus $X_{i,t} \ge X_{i,t'}$ for all $t > t'$.

Let the subscript operator $[\cdot]$ denote the usual order
statistics; let $\lfloor \cdot \rfloor$ denote the floor operator; let
$\lceil \cdot \rceil$ denote the ceiling operator.


\section{System dynamics model}
\label{sec-2}

\subsection{Gravity model}
\label{sec-2-1}

To model the evolution of the disease, we model the probability that
an infected location $i$ infects an uninfected location $j$ at time
point $t$.  We denote this probability as $p_{i,j,t}$.  If we assume
independence across events $i$ infecting $j$ and $i'$ infecting $j$,
the probability of $j$ becoming infected at time $t$ is
\begin{equation*}
  P(X_{j,t} = 1 | X_{\cdot,t-1}) = X_{j,t-1} \bigvee \left[1 - \prod_{i:
      X_{i,t-1} = 1} ( 1 - p_{i,j,t} ) \right].
\end{equation*}

To define $p_{i,j,t}$ we use a form of the gravity model.
\begin{equation*}
  \logit \; p_{i,j,t} = \beta_0 + \beta_1 U_j 
  - \alpha \frac{d_{i,j}}{(c_ic_j)^\nu} - \xi g_j(\tau_{i,t})
  - \rho A_{j,t-1} - \eta A_{i,t-1}
\end{equation*}
where $\tau_{i,t} = \sum_{t' = 0}^t X_{i,t'}$ is the number of time
points $i$ has been infected.  We compare different forms of
$g_j(\cdot)$ discussed later.  The name of the model comes from
$d_{i,j}/(c_i,c_j)^\nu$ which is known as the gravity term.  The
effect of distance between two locations is diminished if the number
of caves in the two locations is large.  We assume as the number of
caves increases, so does the number of bats effectively decreasing the
effect of distance.  The vector $U_j$ is a vector of covariates as
part of the state information.  These covariates include the number of
caves, average winter length, area, species richness, and number of
hibernating bat species.

For model estimation, we use a Bayesian framework.  The priors on the
parameters are as follows $\beta_0 \sim N(0,100)$, $\beta_1 \sim
   N(0,10)$, $\log(\alpha) \sim N(0,1)$, $\log(\nu) \sim N(0,1)$.  The
priors for $\rho$ and $\eta$ will be specified later.  For WNS, the
observed data does not include information about treatments.  This
requires an ad-hoc selection of the true treatment effects.  The prior
distributions during the simulations for these parameters will over
estimate the true effect.

\subsection{Model fit diagnostics}
\label{sec-2-2}

When estimating the posterior distributions, we compared multiple
forms for $g(\cdot)$.
\begin{itemize}
\item ``zero'': $g_j(x) = 0$
\item ``linear'': $g_j(x) = x - 1$
\item ``exp'': $g_j(x) = exp(x - 1) - 1$
\item ``exp\_caves'': $g_j(x) = exp(\frac{\max_k c_k+1}{c_j + 1}(x - 1)) - 1$
\end{itemize}


The table below shows the Bayesian P-value for each of the four
transformations above.  Using an exponential transformation greatly
improves the estimated p-values.  Going forward, we choose the
``exp\(\_\)caves'' transformation as providing the best fit.
 % latex table generated in R 3.1.3 by xtable 1.7-4 package
% Fri Mar 13 11:06:20 2015
\begin{table}[ht]
\centering
\begin{tabular}{rrrrr}
  & zero & linear & exp & exp\_caves \\ 
  \hline
n\_inf & 0.0044 & 0.6837 & 0.6604 & 0.6139 \\ 
  n\_inf\_2007 & 0.0507 & 0.2654 & 0.2942 & 0.3209 \\ 
  n\_inf\_2008 & 0.0000 & 0.4197 & 0.1263 & 0.1175 \\ 
  n\_inf\_2009 & 0.0000 & 0.6249 & 0.3884 & 0.3037 \\ 
  n\_inf\_2010 & 0.0001 & 0.6709 & 0.5682 & 0.4780 \\ 
  n\_inf\_2011 & 0.0013 & 0.6819 & 0.6420 & 0.5804 \\ 
  n\_inf\_2012 & 0.0234 & 0.6813 & 0.6710 & 0.6408 \\ 
  n\_inf\_2013 & 0.1357 & 0.6098 & 0.6757 & 0.6557 \\ 
  mean\_year & 0.7744 & 0.3494 & 0.6573 & 0.6276 \\ 
  mean\_long & 0.9122 & 0.3176 & 0.3735 & 0.4526 \\ 
  mean\_lat & 0.9260 & 0.3189 & 0.4078 & 0.5153 \\ 
  mean\_dist\_from\_start & 0.0484 & 0.6816 & 0.6064 & 0.5132 \\ 
  min\_long & 0.8960 & 0.3169 & 0.3612 & 0.4193 \\ 
  min\_lat & 0.8946 & 0.3170 & 0.3432 & 0.3931 \\ 
  max\_long & 0.0017 & 0.5111 & 0.2298 & 0.2070 \\ 
  max\_lat & 0.0744 & 0.6827 & 0.6384 & 0.5913 \\ 
  max\_dist\_from\_start & 0.1256 & 0.6834 & 0.6495 & 0.5994 \\ 
   \hline
Column Mean & 0.2864 & 0.5186 & 0.4878 & 0.4723 \\ 
  Column Median & 0.0625 & 0.5642 & 0.5280 & 0.4956 \\ 
  Column SD & 0.3794 & 0.1616 & 0.1705 & 0.1508 \\ 
   \hline
\end{tabular}
\caption{Bayesian p-values for postulated models.} 
\end{table}



\section{Structure generation}
\label{sec-3}

\subsection{Structures}
\label{sec-3-1}

\subsubsection{Random Network}
\label{sec-3-1-1}

For each location $i=1,\ldots,L$, $C_i \sim \lbrace U(0,1)\rbrace^2$.
To determine which locations are connected, we must first define some
notation.  Let $E_{i,j} = \|C_{i} - C_{j}\|_2$.  The set of $k$ nearby
locations to location $i$ is $N^{(k)}_i = \lbrace j : E_{i,j} \le
    E_{i,[k+1]} \rbrace$.  Note the detail $[k+1]$ which is a consequence
of $E_{i,i} = 0$.  The adjacency matrix is defined as $\Omega_{i,j} =
    1$ if $i \in N_{j}^{(k)}$ or $j \in N_{i}^{(k)}$.


\begin{figure}[htb]
\centering
\includegraphics[width=0.5\textwidth]{./data/toy/randSample.pdf}
\caption{\label{fig:rand50}A random network with 50 locations}
\end{figure}



\subsubsection{Starred Alley}
\label{sec-3-1-2}

For a starred alley of size $L$, define
\begin{equation*}
m_L= \underset{m > 0}{\arg\max} \; f(m) \mathds{1}_{\lbrace f(m) \le L
\rbrace }
\end{equation*}
where
\begin{equation*}
f(m) = m + \left\lceil \frac{m}{2}
\right\rceil 
\left(\left\lceil \frac{m}{2} \right\rceil
- (m \text{ mod } 2) + 1 \right).
\end{equation*}

Let locations $1,\ldots,m_L$ index the locations on the main
horizontal line.  Next define disjoint sets of locations
$N_i$ for \$i = 1,\ldots,m$_{\text{L}}$.  The cardinality for
$N_i$ is defined as
\begin{equation*}
| N_i | = \left\lfloor \frac{i}{2} \right\rfloor +
\mathds{1}_{\lbrace(m_L - i) < (L - f(m_L))\rbrace}.
\end{equation*}
For each location $j \in N_i$ set $\Omega_{i,j} = \Omega_{j,i} = 1$.
For the first $m_L$ locations, $\Omega_{i,i+1} = \Omega_{i+1,1} = 1$ for
$i = 1,\ldots,m_L$.

To define the coordinates for each location, first set $C_{i} =
    \lbrace i,0 \rbrace$ for $i = 1,\ldots,m_L$.  Then for each $i$,
we set the coordinates for all nodes in $N_i$ according to
\begin{equation*}
  C_{j_{N_i}} = 
  \begin{cases}
    rot(j\pi/(\lceil |N_i|/2 \rceil + 1)) (-1,0) + C_i & j = 1,\ldots,
    \lceil |N_i|/2 \rceil\\
    rot(j\pi/(\lfloor |N_i|/2 \rfloor + 1)) (0,1) + C_i & j = \lceil
    |N_i|/2 \rceil + 1,\ldots,|N_i|
  \end{cases}  
\end{equation*}
where
\begin{equation*}
rot(\theta) = \left[
\begin{matrix}
\cos(\theta) & -\sin(\theta)\\
\sin(\theta) & \cos(\theta)
\end{matrix}
\right].
\end{equation*}

Distances between locations is defined as geodesic distance, but
the edges do not all have length $1$.  Distance between location
$i$ and $j$ is
\begin{equation*}
  d_{i,j} = 
  \begin{cases}
    | i - j | & i,j \in \lbrace 1,\ldots,m_L \rbrace\\
    | i - k | + .9 & j \in N_k \text{ and } i \in \lbrace 1,\ldots,m_L
    \rbrace\\
    | k - j | + .9 & i \in N_k \text{ and } j \in \lbrace 1,\ldots,m_L
    \rbrace\\
    | k - \ell| + 1.8 & i \in N_k \text{ and } j \in N_\ell \text{ and
      } k,\ell \in \lbrace 1,\ldots,m_L \rbrace
  \end{cases}
\end{equation*}


\begin{figure}[htb]
\centering
\includegraphics[width=0.5\textwidth]{./data/toy/alleySample.pdf}
\caption{\label{fig:alley50}The starred alleyway network with 50 locations}
\end{figure}



\subsubsection{Ring}
\label{sec-3-1-3}

Divide the number of locations $L$ into two groups.  Define $L_1 =
    L - L_2$ where $L_2 = \lceil L*0.05 \rceil$.  Let $\theta =
    2\pi/(L_1+1)$ and $r = 1/(2\sin(\theta/2))$.  Place location $1$
at $C_1 = \lbrace 0,r \rbrace$.  Given location $i-1$ at
$C_{i-1}$, place location $i$ at $C_i = rot(\theta) C_{i-1}$ for
$i = 2,\ldots,L_1$.  Note that $C_{1} \cdot C_{L_1} = \|C_{1}\|
    \|C_{L_1}\| \cos(2\theta)$.

With the first $L_1$ locations defined, we need to define the last
$L_2$ locations.  Place location $L_1 + i$ at $C_{L_1 + i} =
    rot(\theta/L_2)C_{L_1 + i - 1}$ for $i = 1,\ldots,L_2$.  Note that
$C_{L} \cdot C_{L_1} = \|C_{L}\| \|C_{L_1}\| \cos(\theta)$ and $C_{1}
    \cdot C_{L} = \|C_{1}\| \|C_{L}\| \cos(\theta)$.

To define the adjacency matrix, $\Omega_{i,i+1} = \Omega_{i+1,i} = 1$ for $i =
    1,\ldots,L-1$ and $\Omega_{1,L} = \Omega_{L,1} = 1$.

Defining distance is defined as the minimum arc length between two
locations.
\begin{equation*}
  d_{i,j} = r\cos^{-1}\left(\frac{C_{i} \cdot
      C_{j}}{\|C_{i}\|\|C_{j}\|}\right)
\end{equation*}





\begin{figure}[htb]
\centering
\includegraphics[width=0.5\textwidth]{./data/toy/ringSample.pdf}
\caption{\label{fig:ring50}The ring network with 50 locations}
\end{figure}




\subsubsection{Grid}
\label{sec-3-1-4}

First define $L_1$ to be the number of columns and $L_2$ to be the
number of rows.  These values are defined by
\begin{equation*}
  \lbrace L_1, L_2 \rbrace = \underset{
    \begin{subarray}{c}
      \ell_1,\ell_2 > 0\\
      \ell_1\ell_2 = L\\
      \ell_1 \le \ell_2
    \end{subarray}
  }{\arg\min} \quad |\ell_1 - \ell_2|
\end{equation*}

Location $i$ is positioned at $C_i = \lbrace (i \text{ mod } L_2), \lfloor
    i/L_2 \rfloor \rbrace$.  The adjacency matrix is defined as $\Omega_{i,j} =
    1$ if $\|C_i - C_j\| = 1$.

\begin{figure}[htb]
\centering
\includegraphics[width=0.5\textwidth]{./data/toy/gridSample.pdf}
\caption{\label{fig:grid50}The grid network with 50 locations}
\end{figure}



\subsubsection{Bowtie}
\label{sec-3-1-5}

The bowtie combines two grid networks with a random network in
between.  Spread of the disease begins in one grid and travels through
the random network and into the other grid.

For a bowtie network of size $L$, create one random network of size
$L_R = \lceil 0.1 L \rceil + ((L - \lceil 0.1 L \rceil) \text{ mod }
    2)$ and two grid networks of size $L_G = (L - L_R)/2$.  Let $i_{R}$ be
the $i^{th}$ location in the random network and $i_{G_j}$ be the
$i^{th}$ location in the $j^{th}$ grid.

To connect the three networks, we need to first identify which nodes
in each network will connect to one of the other two networks.  We
will begin with the first grid.  First, define $L_1$ and $L_2$ to be
defined in the same way as in the grid description.  Let $M_{G} =
    \lceil L_2/3 \rceil$ be the number of points to connect from the grid
networks and $M_{R} = \lceil L_R/5 \rceil$.  Define the following four
sets of locations
\begin{equation*}
  \begin{array}{rcl}
    \Lambda_{G_1} & = & \lbrace L_G -
                        2L_1M_G + kL_1 : k = 1,\ldots,M_G\rbrace\\
    \Lambda_{G_2} & = & \lbrace L_G -
                        2L_1M_G + 1 + (k-1)L_1 : k =
                        1,\ldots,M_G\rbrace\\
    \Lambda_{R_1} & = & \lbrace i : C_{i_R,1} \le C_{[M_R]_{R},1} \rbrace\\
    \Lambda_{R_2} & = & \lbrace i : C_{i_R,1} \ge C_{[L_R - M_R + 1]_{R},1} \rbrace.
  \end{array}
\end{equation*}

To define the adjacency matrix, combine the adjacency matrices from
the three separate networks as a block diagonal matrix.  To define the
additional edges, set $\Omega_{i_{G_1},j_R} = \Omega_{j_R,i_{G_1}} =
    1$ for all $i \in \Lambda_{G_1}$ and $j \in \Lambda_{R_1}$ to connect
the first grid to the random network.  Then set $\Omega_{i_{G_2},j_R}
    = A_{j_R,i_{G_2}} = 1$ for all $i \in \Lambda_{G_2}$ and $j \in
    \Lambda_{R_2}$ to connect the second grid to the random network.

Now that the adjacency marix is defined, we need to finish the
network.  The last step is scaling and adjusting the coordinates from
the three networks.  We define this as an ordered list of operations.
\begin{enumerate}
  \item For $i=1,\ldots,L_R$: $C_{i_{R}} = C_{i_{R}}/2$.
  \item Define $K = (\max_i C_{i_{R},1} - \min_i C_{i_{R},1})/2$.
  \item For $i=1,\ldots,L_R$: $C_{i_{R},1} = C_{i_{R},1} -
  \min_j C_{j_{R},1} + \max_j C_{j_{G_1},1} + K$
  \item For $i=1,\ldots,L_G$:
  $C_{i_{G_2},1} = C_{i_{G_2},1} - \min_j C_{j_{G_2},1} + \max_j
  C_{j_{R},1} + K$.
\end{enumerate}


\begin{figure}[htb]
\centering
\includegraphics[width=0.5\textwidth]{./data/toy/bowtieSample.pdf}
\caption{\label{fig:bowtie50}The bowtie network with 50 locations}
\end{figure}


\subsubsection{Scalefree}
\label{sec-3-1-6}

The scalefree network is designed so that some locations are highly
connected while others are only connected by a couple edges.  This
type of network was originally desigined to mimic the network of
webpages.  The internet houses many websites that are highly connected
(e.g. Google, Yahoo, etc.) while the majority are not.

To generate the network of size $L$, we will define the process via
induction.  Define $\Omega^{(i)}$ to be the adjacency network on
iteration $i = 1,\ldots,L$.  A network of size $1$ is the trivial
network.  Given a network of size $N-1$, place the additonal $N^{th}$
location adjacent to an existing location $i \in \lbrace
    1,\ldots,N-1\rbrace$ with probability
\begin{equation*}
  P(\Omega^{(N)}_{N,i} = 1 | \Omega^{(N-1)}) = 
  \frac{\sum_{j!=i} \Omega^{(N-1)}_{j,i}}{\sum_{j,k \;:\; j > k} \Omega^{(N-1)}_{j,k}}.
\end{equation*}

Generating a scale-free network in this fashion results in a
Barabasi-Network.  The defining feature of a scale-free network is the
proportion of nodes with $k$ edges is proportional to $k^{-\gamma}$
for some $\gamma$.  In a Barabasi-Network $\gamma = 3$.

Given the adjacency matrix, we use the Fruchterman-Reingold algorithm
to position the nodes in a two dimensional plane.


\begin{figure}[htb]
\centering
\includegraphics[width=0.5\textwidth]{./data/toy/scalefreeSample.pdf}
\caption{\label{fig:scalefree50}The scalefree network with 50 locations}
\end{figure}




\subsection{Covariates}
\label{sec-3-2}

The covariates $U_i$ are multivariate normal $MVN_p(\mu,\Sigma)$
where $\mu_i = 2\widetilde{C}_{i,1}^2 + 1\widetilde{C}_{i,2}$ and
$\widetilde{C}$ are the centered and scaled coordinates.  The
covariance matrix has the structure $\Sigma_{ip + s, jp + t} = \psi
   e^{(-\zeta E_{i,j} - \phi |s-t|)}$.  Thus, the variance of the
individual covariates are $1$ and the correlation decays as
$E_{i,j}$ increases and as $|s-t|$ increases.

We need the number of caves for each location to compute the
gravity term in the gravity model.  Define $c_i = \lfloor U_{i,1} -
   U_{[1],1} \rfloor + 1$ to be the mass of location $i$.


\section{Simulation Experiment}
\label{sec-4}

\subsection{Implementation details}
\label{sec-4-1}

For each structure, we calibrate the generative model to have
certain characteristics.  We use the estimated posterior
distributions from the observed WNS data as a starting point to
insure realistic effect sizes.

Using the generative model, we are able to simulate the spread of
the disease on the structure.  The simulation runs for $T$ time
points at which the final reward is recorded.  This simulation
experiment aims at comparing the expected reward under different
intervention strategies.  The expectred reward is estimated using
Monte Carlo integration.

\subsubsection{Setting the generative model}
\label{sec-4-1-1}

Let $\lbrace \widetilde{\beta}_0, \widetilde{\beta}_1^\T,
    \widetilde{\alpha}, \widetilde{\nu}, \widetilde{\xi} \rbrace$ be
the estimated posterior mean from the observed WNS data.  Since
treatments are not included in the observed data, we exlude
$\widetilde{\rho}$ and $\widetilde{\eta}$ and define these below.
For the above generated structures, we make two changes to the
gravity model to force the network to have a larger impact on the
dynamics of the disease.  First, we scale all paramters excluding
$\nu$ by a constant $\omega$.  Second, we raise $d_{i,j}$ to a
power $h(\omega,\widetilde{\alpha},\widetilde{\nu})$.  The
generative model for the experiment is
\begin{equation*}
  \logit \; p_{i,j,t} = \omega\widetilde{\beta}_0 + \omega\widetilde{\beta}_1 U_j 
  - \omega\widetilde{\alpha}
  \frac{d_{i,j}^{h(\omega,\widetilde{\alpha},\widetilde{\nu})}}{(c_ic_j)^{\widetilde{\nu}}}
  - 
  \omega\widetilde{\xi} g_j(\tau_{i,t})
  - \omega\widetilde{\rho} A_{j,t-1} - \omega\widetilde{\eta} A_{i,t-1}
\end{equation*}
where 
\begin{equation*}
  h(\omega,\widetilde{\alpha},\widetilde{\nu}) =
  \frac{\log\left(\frac{\overline{c}^{2\omega\widetilde{\nu}}\log(2.0)}
      {\omega\widetilde{\alpha}}
      + 1 \right)}{\log(2.0)}.
\end{equation*}
Take three hypothetical locations $i,j,j'$.  Location $i$ is
infected and $j,j'$ are not.  Assume no intervention and locations
are all identical with $c_i,c_j,c_{j'} = \overline{c}$ where
$\overline{c} = \frac{1}{L}\sum_i c_i$.  The only differences
between the locations is $d_{i,j} = 1$ and $d_{i,j'} = 2$.  The
form of $h$ comes from setting
\begin{equation*}
  \frac{\frac{p_{i,j,0}}{1-p_{i,j,0}}}{\frac{p_{i,j',0}}{1-p_{i,j',0}}}
  = 2.0
\end{equation*}
We set $\omega$ such that the expected reward under no
intervention is $0.3$.

The final component of the generative model is defining
$\widetilde{rho},\widetilde{eta}$.  Again, take hypothetical
locations $i,j$ where $i$ is infected and $j$ is not.  Assume both
receive intervention, $c_i = c_j = \overline{c}$, and $d_{i,j} =
    \underset{i,j\in\mathcal{L}} \min d_{i,j}$.  They treatment
effects are set such that
\begin{equation*}
  \logit(p_{i,j,0}) = 0.005
\end{equation*}


\subsubsection{Details of each trajectory}
\label{sec-4-1-2}

The simulations begin at time point $0$ and end at $T=15$.  Observed
data from WNS has data from $8$ years.  When simulating the spread
of the disease under intervention, treatments are not given until
time point $8$ to mimic the observed data.

At time $t \ge 8$, the posterior distributions of the indexing
parameters in the system dynamics model are estimated.  Using these
estimated distributions, the simulataneous perturbation algorithm is
used to maximize the estimated posterior mean reward at time $T$
with respect to the priority score weights.  The reward funtion at
time $t$ is the current proportion of locations not infected,
$Y^t(\pi) \triangleq \frac{1}{L}\sum_i 1 - X_{i,t}$.

At $t = 9$, we add one component to the simulation.  At time this an
online tuning is run for simultaneous perturbation to adaptively
improve the optimization performance.

The prior for the treatment effects are set to be optimistic.  They
are normal with mean $4$ times as large as the true effect and
variance $1$.

[An algorithm sketch would be useful here]


\section{Results}
\label{sec-5}

\subsection{Simultaneous perturbation experiment}
\label{sec-5-1}

To calibrate the simultaneous perturbation algorithm, we ran a full
factorial experiment over the following levels.
\begin{itemize}
  \item $A \in \lbrace 30, 50 \rbrace $
  \item $B \in \lbrace 1, 10 \rbrace $
  \item $C \in \lbrace 2, 5 \rbrace $
  \item $L \in \lbrace 1, 1.25 \rbrace $
  \item $T \in \lbrace 1, 2 \rbrace $
\end{itemize}

For each of the $32$ combinations, the table below shows the mean and
standard deviation of the value across all replications of that factor
combination.  The results are sorted by the mean value.

 % latex table generated in R 3.1.3 by xtable 1.7-4 package
% Fri Mar 13 11:06:20 2015
\begin{table}[ht]
\centering
\begin{tabular}{rrrcccccr}
  \hline
valueMean & valueSd & timeMean & Afac & Bfac & Cfac & Tfac & Lfac & combo \\ 
  \hline
0.554 & 0.016 & 2.757 & - & + & + & - & - & 12 \\ 
  0.560 & 0.021 & 2.741 & + & - & + & - & - & 20 \\ 
  0.565 & 0.030 & 2.700 & + & + & + & - & - & 28 \\ 
  0.565 & 0.017 & 2.789 & + & - & - & - & - & 16 \\ 
  0.568 & 0.019 & 2.817 & - & - & + & - & - & 4 \\ 
  0.569 & 0.009 & 2.746 & + & + & - & - & - & 24 \\ 
  0.574 & 0.014 & 0.909 & + & - & + & - & + & 21 \\ 
  0.574 & 0.013 & 0.920 & + & - & - & - & + & 17 \\ 
  0.576 & 0.017 & 0.915 & - & - & + & - & + & 5 \\ 
  0.578 & 0.029 & 2.848 & - & - & - & - & - & 0 \\ 
  0.578 & 0.014 & 0.931 & - & + & - & - & + & 9 \\ 
  0.581 & 0.032 & 0.898 & + & + & + & - & + & 29 \\ 
  0.581 & 0.017 & 2.825 & - & + & - & - & - & 8 \\ 
  0.584 & 0.014 & 0.917 & + & + & - & - & + & 25 \\ 
  0.584 & 0.019 & 0.920 & - & + & + & - & + & 13 \\ 
  0.587 & 0.022 & 0.942 & - & - & - & - & + & 1 \\ 
  0.631 & 0.023 & 0.791 & + & - & + & + & + & 23 \\ 
  0.649 & 0.026 & 0.789 & - & + & + & + & + & 15 \\ 
  0.650 & 0.012 & 0.821 & - & - & + & + & + & 7 \\ 
  0.658 & 0.017 & 0.811 & + & + & + & + & + & 31 \\ 
  0.662 & 0.011 & 0.729 & + & - & - & + & + & 19 \\ 
  0.664 & 0.029 & 0.773 & - & - & - & + & + & 3 \\ 
  0.667 & 0.016 & 0.721 & + & + & - & + & + & 27 \\ 
  0.667 & 0.015 & 0.740 & - & + & - & + & + & 11 \\ 
  0.668 & 0.030 & 2.072 & + & + & + & + & - & 30 \\ 
  0.670 & 0.040 & 1.964 & + & + & - & + & - & 26 \\ 
  0.671 & 0.028 & 2.124 & + & - & - & + & - & 18 \\ 
  0.675 & 0.018 & 2.040 & - & + & + & + & - & 14 \\ 
  0.686 & 0.024 & 2.094 & - & - & - & + & - & 2 \\ 
  0.688 & 0.025 & 1.973 & - & + & - & + & - & 10 \\ 
  0.700 & 0.020 & 2.108 & - & - & + & + & - & 6 \\ 
  0.710 & 0.024 & 2.078 & + & - & + & + & - & 22 \\ 
   \hline
\end{tabular}
\caption{Results from the simultaneous perturbation experiment.} 
\end{table}


\subsection{Agent jitter and iterations experiment}
\label{sec-5-2}

\subsection{Toy Structures}
\label{sec-5-3}

\subsection{WNS}
\label{sec-5-4}
% Emacs 24.4.1 (Org mode 8.2.10)
\end{document}